\documentclass{article}  % Tipo di documento
\usepackage[utf8]{inputenc} % Per caratteri accentati
\usepackage[italian]{babel} % Lingua italiana
\usepackage{amsthm}
\usepackage{amssymb}
\usepackage{amsmath}  % simboli matematici avanzati
\usepackage{xcolor} % Per i colori
\usepackage{titlesec} % Per personalizzare i titoli
\usepackage{tikz}
\usetikzlibrary{mindmap,trees}
\usepackage[most]{tcolorbox}
\usepackage{subcaption}  % per avere subfigure
\tcbuselibrary{theorems}
\usepackage{tikz}
\usetikzlibrary{automata, positioning, arrows}
\tcbuselibrary{breakable}
\usepackage{graphicx}
\usepackage[table]{xcolor} % da mettere nel preambolo
\usepackage{mathrsfs} % https://www.ctan.org/pkg/mathrsfs
\graphicspath{ {./media/} }
\usepackage{centernot}
\usepackage[hidelinks]{hyperref}  

\newtcbtheorem[no counter]{theorem}{Teorema}%
{colback=blue!5, 
colframe=blue!50!black, 
fonttitle=\bfseries,
    breakable,            % permette di spezzare il box su più pagine
    enhanced,
    break at=0pt}{}

\theoremstyle{definition}
\newtheorem{definition}{Definizione}[section]

% Imposto colore delle subsection
\titleformat{\subsection}
  {\normalfont\large\color{red}} % stile del titolo
  {\thesubsection}{1em}{} % numerazione e spaziatura

% Definiamo un nuovo ambiente per gli esempi
\newtcolorbox{esempio}[1][]{
    colback=white,       % colore di sfondo
    colframe=gray,       % colore del bordo
    fonttitle=\bfseries,
    title=#1,
    boxrule=0.5pt,       % spessore del bordo
    arc=4pt,             % angoli arrotondati
    left=4pt, right=4pt, top=4pt, bottom=4pt,
	breakable,            % permette di spezzare il box su più pagine
    enhanced,
    break at=0pt
}

\newtcolorbox{esercizio}[1][]{
    colback=white,       % colore di sfondo
    colframe=green!60!black,       % colore del bordo
    fonttitle=\bfseries,
    title=#1,
    boxrule=0.5pt,       % spessore del bordo
    arc=4pt,             % angoli arrotondati
    left=4pt, right=4pt, top=4pt, bottom=4pt,
    breakable,            % permette di spezzare il box su più pagine
    enhanced,
    break at=0pt
}

\newtcolorbox{osservazioni}[1][]{
    colback=white,       % colore di sfondo
    colframe=yellow!80!orange,       % colore del bordo
    fonttitle=\bfseries,
    title=#1,
    boxrule=0.5pt,       % spessore del bordo
    arc=4pt,             % angoli arrotondati
    left=4pt, right=4pt, top=4pt, bottom=4pt,
    breakable,            % permette di spezzare il box su più pagine
    enhanced,
    break at=0pt
}

% Creo un nuovo ambiente "ragionamento" senza quadratino
\newenvironment{ragionamento}[1][]
  {\begin{proof}[Ragionamento#1]\renewcommand{\qedsymbol}{}\normalfont}
  {\end{proof}}

\title{Statistica}
\author{Ede Boanini}
\date{\today}

\begin{document}
\maketitle
\tableofcontents % genera automaticamente l’indice
\newpage
%%%%%%%%%%%%%%%%%%%%%%%%%%%%%%%%%%%%%%%%%%%%%%%%%%%%%%%%%%%%%%%%%%%%%%%%%%%%%%%%%%%%%%%%%%%%%%%%%%%%%%
\section{Introduzione}
\subsection{Classificazione delle Variabili}
\begin{center}
	\begin{tikzpicture}[
			level 1/.style={
					sibling distance=50mm,
					level distance=15mm,
					every node/.append style={font=\small} % qui riduco il font dei figli
				},
			level 2/.style={
					sibling distance=25mm, % distanza tra i nodi di livello 2
					level distance=15mm    % distanza verticale
				},
			every node/.style={
					rectangle, draw, rounded corners,
					align=center,
					top color=orange!60,
					bottom color=orange!10
				}
		]
		\node {Variabili}
		child {node {Quantitative (numeriche)}
				child {node {Discrete \\ $n \in \mathbb{N}$}}
				child {node {Continue \\ $n \in \mathbb{R}$}}
			}
		child {node {Qualitative (categoriche)}
				child {node {Ordinali}}
				child {node {Nominali}}
			};
	\end{tikzpicture}
\end{center}
Differenza tra ordinali e nominali:
\begin{itemize}
	\item \textbf{Ordinali:} categorie che hanno un ordine, puoi solo dire se un valore è minore o maggiore rispetto ad un altro.
	      \footnotesize
	      \textit{
		      \begin{itemize}
			      \item Livello di istruzione: elementare $<$ media $< \cdots$
			      \item Grado di soddisfazione: nullo $<$ basso $<$ medio $< \cdots$
			      \item Classifica di una gara: quinto$<$quarto$< \cdots$
			      \item Matricola: 17345 $<$ 17346 $< \cdots$
		      \end{itemize}}
	\item \textbf{Nominali:} categorie che non hanno un ordine.
	      \footnotesize
	      \textit{
		      \begin{itemize}
			      \item Colore occhi: blu, verdi, marroni, $\cdots$
			      \item Genere: M, F
			      \item Marche auto: Toyota, Ford, $\cdots$
			      \item Nazionalità: Giapponese, Italiano, $\cdots$
		      \end{itemize}
	      }
\end{itemize}
\subsection{Distribuzioni di Frequenza}
È una tabella che contiene modalità e frequenze.
\begin{center}
	\includegraphics[width=0.5\linewidth]{dist-freq.png}
\end{center}
\subsubsection{Tipi di Frequenza}
\begin{enumerate}
    \item \textcolor{red}{Frequenza assoluta:} numero di ripetizioni di una certa modalità (es: quanti studenti hanno preso 28 all'esame)
    \begin{center}
        $freq_{assoluta}=f_i$
    \end{center}
    \item \textcolor{red}{Frequenza relativa:} 
    \begin{center}
        $freq_{relativa}=\frac{f_i}{N}$
    \end{center}
    \item \textcolor{red}{Frequenza percentuale:} 
    \begin{center}
        $freq_{\%}=\frac{f_i}{N}\cdot 100$ \\oppure\\
        $freq_{\%}= freq_{relativa} \cdot 100$
    \end{center}
    \item \textcolor{red}{Frequenza cumulata:} somma progressiva delle frequenze assolute o relative.
        \[ freq_{cumulataAssoluta}= \sum_{i=1}^{n} f_i \]
        \[ freq_{cumulataRelativa}= \sum_{i=1}^{n} \frac{f_i}{N} = \sum_{i=1}^{n} freq_{relativa_i}\]
    \item \textcolor{red}{Frequenza cumulata percentuale:}
        \[ freq_{cumulataAssoluta\%}= \sum_{i=1}^{n} f_i \cdot 100 \]
        \[ freq_{cumulataRelativa\%}= \sum_{i=1}^{n} \frac{f_i}{N} \cdot 100 = \sum_{i=1}^{n} freq_{relativa_i} \cdot 100 \] 
\end{enumerate}


\break
\section{Statistica Descrittiva}
\section{Probabilità}
\section{Indipendenza e Probabilità Condizionata}
\section{Variabili Casuali}
\subsection{Famiglie Parametriche}
\section{Inferenza Statistica}
\subsection{Stima Puntuale}
\subsection{Stima Intervallare}
\subsection{Verifica delle Ipotesi}
%%%%%%%%%%%%%%%%%%%%%%%%%%%%%%%%%%%%%%%%%%%%%%%%%%%%%%%%%%%%%%%%%%%%%%%%%%%%%%%%%%%%%%%%%%%%%%%%%%%%%%
\end{document}
